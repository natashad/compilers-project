\title{CSC488H1S - Assignment 2: Design Document.}
\author{
        Group 2\\
        g0dalaln - Natasha Dalal\\
        g0getter - Tony (Hao) Cheng\\
        g0faizan - Faizan Rashid\\
        g0alimuh - Showzeb Ali\\
        g3ksingh - Karandeep Basi\\
}
\date{\today}

\documentclass[12pt, a3paper]{article}
\usepackage[a3paper]{geometry}

\begin{document}
\maketitle

\paragraph{}
When designing the grammar, I started by simply following the language specification, correcting obvious problems as I went.
The first real variation from the given language specification was splitting up grammar rules that could lead to multiple parse trees.
For instance, $statement \rightarrow statement \ statement.$ In order to disambiguate cases like these, I split up rules with two recursive references into another rule.
Therefore with the example of $statement$, I created $statementList \rightarrow statement \ | \ statement \ statementList.$ Then, wherever there were references to $statement$ in the
reference language rules, I replaced them with $statementList$. A similar procedure was followed for $declaration$ (leading to $declarionList$), $variablenames$ (leading to $variableNamesList$),  $output$ and $input$ (leading to $outputList$ and $inputList$), $arguments$ (leading to $argumentsList$), and $paramaters$ (leading to $parametersList$).

\paragraph{}
Another deviation from the reference grammar involved replacing all references to variablenames, parameternames, functionnames, procedurenames, etc... with the single $IDENT$ in order to prevent reduce-reduce conflicts.

\paragraph{}
The part of the grammar I found most challenging was the expressions. Specifically since expressions needed to be defined in a way to allow for operator precedence and associatively.
The way that I handled this was by starting at a base form that I called $baseExpression$. This base form defined some of the  most basic forms of expressions (those not including operators): Terminals, function calls, expressions in parantheses etc.... From there, I built up at every step starting with the operator of highest precedence, including in its rules a rule to generate the previous type and then a rule using the previous type with the current operator. For example: the unary minus has highest precedence and so I did $UnaryExpression \rightarrow BaseExpression \ | \  MINUS\  BaseExpression$. In this way, I built up all the way to the $or$ operator which had the lowest precedence. $OrExpression$ was therefore inclusive of all other generatable expressions, as well as expressions separated by an $or$. Since $and$ was the operator with precedence directly above $or$, $orExpression$ was defined as $andExpression \ | \ orExpression \ OR \ andExpression$. The use of $orExpression$ as the fist non-terminal in the second rule for $orExpression$ was to allow for associativity and will be explained below. I then used $OrExpressions$ as my $finalExpression$ which is used wherever the reference language makes reference to expression. This $finalExpression$ was also used in $baseExpressions$ within parentheses.

\paragraph{}
An important note is that in order to maintain associativity from Left to Right for multiplication, division, addition, subtraction, and, or (and also not and unary minus -- though the direction of associativity was not relevant there) My rules for those have a recursive part as the left most non-terminal on the RHS of the rule instead of the operator expression above it in the precedence hierarchy. This ensures the left to right associativity. For example: $andExpression \rightarrow notExpression \ |  \ andExpression \ AND \ notExpression$ instead of $notExpression \ AND \ notExpression$. This would allow for something like a AND b AND c which would be parsed as something like ((a AND b) AND c). Such a recursive rule was not used for the equality type of operators ($=, <, >, <=, >=, not=$) as those rules were not meant to be associative.
\end{document}