\title{CSC488H1S - Assignment 2: Design Document.}
\author{
        Group 2\\
        g0dalaln - Natasha Dalal\\
        g0getter - Tony (Hao) Cheng\\
        g0faizan - Faizan Rashid\\
        g0alimuh - Showzeb Ali\\
        g3ksingh - Karandeep Basi\\
}
\date{\today}

\documentclass[12pt, a3paper]{article}
\usepackage[a3paper]{geometry}

\begin{document}
\maketitle

\paragraph{}
When designing the grammar, I started by just simply following the language specification, correcting obvious problems as I went.
The first real variation from the given specification was splitting up grammar rules that could lead to multiple parse trees.
For instance, $Statement \rightarrow Statement \ Statement.$ In order to disambiguate cases like these, I split up rules with two recursive references into another rule.
Therefore with the example of Statement, I created $StatementList \rightarrow Statement \ | \ Statement \ StatementList.$ Then wherever there were references to Statement in the
reference language rules, I replaced them with $StatementList$.

\paragraph{}
Another deviation from the reference grammar involved replacing all references to variablenames, parameternames, functionnames, prcedurenames, etc... with $IDENT$.

\paragraph{}
The part of the grammar I found most challenging was the expressions. Specifically, expressions needed to be defined in a way to allow for operator precedence.
The way that I handled this was by starting at a base form that I called $baseExpression$. This base form defined some of the  most basic forms of expressions: Terminals, argument lists, expressions in parantheses etc.... From there, I built up at every step starting with the operator of highest precedence, including in its rules a rule to generate the previous type and then a rule using the previous type with the current operator. For example: the unary minus has highest precedence and so I did $UnaryExpression \rightarrow BaseExpression \ | \  MINUS\  BaseExpression$. In this way, I built up all the way to $OR$ which had the lowest precedence. Therefore, OrExpression was inclusive of all other generatable expressions, as well as such expressions with an $OR$ symbol between them. I then used $ORExpressions$ as my $finalExpression$ which is used wherever the reference langugae makes reference to expression. This $finalExpression$ was also used in $baseExpressions$ within parentheses.

\paragraph{}
In order to maintain associativity from Left to Right for multiplication, addition etc... My rules for those have a recursive part as the left most non-terminal on the RHS of the rule. This ensures the left to right associativity.
\end{document}